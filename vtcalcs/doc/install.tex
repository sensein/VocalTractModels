\chapter{Installation and User guide}

\section{Downloading the program}
The program may be downloaded from the following URLs:
\begin{itemize}
\item \href{http://cns-web.bu.edu/~satra/pub/vtcalcs.tgz}
{http://cns-web.bu.edu/~satra/pub/vtcalcs.tgz - for Unix}
\item \href{http://cns-web.bu.edu/~satra/pub/vtcalcs.zip}
{http://cns-web.bu.edu/~satra/pub/vtcalcs.zip - for MS Windows}
\end{itemize}

\section{Installation}
The program can be installed in any directory. However the
directory structure has to be maintained when unpacking.

\begin{enumerate}
\item Choose a directory for installation.
\item Copy the archived file into the directory.
\item Unzip the file while \em maintaining the directory structure \em. On
UNIX systems one may use the command: \begin{center} gunzip -c
vtcalcs.tgz $|$ tar xf \end{center}  On Microsoft Windows systems
either use an unzip program like Winzip or use pkunzip with the
command:
\begin{center} pkunzip -d vtcalcs.zip
\end{center} Most linux systems support the following command
directly:
\begin{center} tar zxf vtcalcs.tgz \end{center}
\item A directory called vtcalcs will be created with three
subdirectories: data, doc, and src. `data' contains some necessary
data files for running the program. `doc' contains this document.
`src' contains the source code for compiling the mex functions.
\item If you are installing it on a system for which mex files
have not been provided you need to create the mex files. This is
described in the next section.
\end{enumerate}

\section{Creating the mex files}
Creating the mex files requires that you have the Matlab compiler
installed properly. For installing the Matlab compiler please refer to
the documentation provided by Mathworks. On a MS Windows system this
typcially requires both the Matlab compiler as well as an additonal
compiler such as Microsoft Visual C++. Matlab v5.3 works fine however
v5.2 and below is known to have problems on newer Linux systems and
can result in segmentation violations. The following URL should help
solve the problem:
\begin{center}
\href{http://www.mathworks.com/support/solutions/v5/11129.shtml}
{http://www.mathworks.com/support/solutions/v5/11129.shtml}
\end{center}

\begin{enumerate}
\item Start Matlab
\item Change to the source directory.
\item At the Matlab prompt type Makefile('unix') or
Makefile('windows').The function is case sensitive. If the
compiler has been set up properly, the call should have created
mex files for you specific platform.
\item Copy or move the mex files from the `src' directory to the vtcalcs
directory. The files will have different extensions on different
platforms. Some common platforms and extensions are listed below.
Please refer to the compiler documentation for any other
platforms.\\
\begin{center}
\begin{tabular}{|c|c|}
\hline
   Platform    & Extension \\
\hline
 Windows 9x/NT &   .dll    \\
\hline
     Linux     &  .mexlx   \\
\hline
  Solaris  &   .mex4   \\
\hline
\end{tabular}
\end{center}
\end{enumerate}

\section{Using the program}
To start the program open Matlab in the same directory where the
m-file vtcalcs.m is. On Unix this can be accomplished by starting
Matlab in the directory containing the file. On Windows one can
change the directory from within Matlab by using the `cd' command.
Once you are in the directory start the program by typing vtcalcs
at the Matlab prompt. This launches the user interface (UI) for
the program.

The UI has three menu options relevant to running the program.
These are:
\begin{itemize}
\item VT Calculation
\item Tract configuration
\item Physical constants
\end{itemize}

\subsection{VT Calculation from Models}
The current version provides 5 different methods of calculating
the transfer function. When selected it provides the user with the
5 options. The sixth option is currently not available and has
been disabled. Choosing any of the other options opens a dialog
box with rather intuitive controls. Most of the pushbuttons with
numeric values on them popup dialogs which allow those values to
be changed. Some push buttons toggle states and some perform a
particular action (eg. synthesize). The sliders change values
continuously. The allowable range of the values are provided in
Appendix A.

\subsection{Changing Tract Configuration}
When this menu option is selected a dialog box with four
pushbuttons pop up. Clicking on any of the buttons toggles its
state and the current state is displayed on the button.

\subsection{Changing Physical Constants}
When this menu option is selected a dialog box with five
pushbuttons pop up. Clicking on any of the buttons opens a dialog
box where one can enter a new value. If no value is entered, the
current value is retained. This dialog also keeps the entered
value within range of possibly allowed values.

\section{Registered Trademarks}
\begin{itemize}
\item Matlab is a registered trademark of Mathworks Inc.
\item Microsoft Visual C++, Microsoft Windows, MS Windows, Windows 9x/NT are registered trademarks
of Microsoft corporation.
\item Solaris is a registered trademark of Sun Microsystems.
\end{itemize}
